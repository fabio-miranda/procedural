% This text is proprietary.
% It's a part of presentation made by myself.
% It may not used commercial.
% The noncommercial use such as private and study is free
% May 2007
% Author: Sascha Frank 
% University Freiburg 
% www.informatik.uni-freiburg.de/~frank/
%
% 
\documentclass{beamer}
\usepackage[latin1]{inputenc}
\usepackage[brazil]{babel}



%Temas
\usetheme[compress]{Dresden}
%\useoutertheme[subsection=false]{smoothbars}
%\useinnertheme[shadow]{rounded}
\useinnertheme{rectangles}


\makeatletter
\beamer@theme@subsectionfalse
\makeatother

%\usebackgroundtemplate{\includegraphics[width=\paperwidth]{img/fundo}}





\begin{document}
\title[Gera��o Procedural de Terrenos Infinitos na GPU]{Desenvolvimento de um Arcabou�o para a Gera��o Procedural e Visualiza��o de Terrenos em Tempo-Real}
\author[F�bio Miranda - Apresenta��o final - POC II]{F�bio Markus Nunes Miranda\\Orientador: Prof. Luiz Chaimowicz\\Co-Orientador: Carl�cio Cordeiro}
\institute[Universidade Federal de Minas Gerais]{Departamento de Ci�ncia da Computa��o\\Universidade Federal de Minas Gerais}
\date[\today]{Apresenta��o final - POC II}

\begin{frame}
\titlepage
\end{frame}


\AtBeginSection[]
{
\begin{frame}
\frametitle{Sum�rio}
\setcounter{tocdepth}{1}
\tableofcontents[currentsection]
\setcounter{tocdepth}{2}
\end{frame}
}






\section{Motiva��o}

\subsection{Motiva��o}

\begin{frame}\frametitle{Motiva��o} 
\begin{itemize}
	\item Atualmente, h� uma necessidade de se criar modelos 3D cada vez maiores e com grande n�vel de detalhe.
	\item Por�m, quanto maior e mais detalhado o modelo, mais tempo ter� que ser gasto por um modelador para faz�-lo.
\end{itemize}	
\end{frame}

\subsection{Gera��o Procedural}

\begin{frame}\frametitle{O que � gera��o procedural?} 
\begin{itemize}
	\item Gera��o procedural � um termo gen�rico para descrever algoritmos que determinam caracter�sticas de efeitos ou modelos.
	\item H� diversos tipos de t�cnicas e algoritmos, cada um aplicado a uma determinada �rea:
	\begin{itemize}
		\item L-System: gera��o de �rvores e cidades.
		\item Fractais e Perlin Noise: gera��o de terrenos e texturas
	\end{itemize}
\end{itemize}
\end{frame}


%\begin{frame}\frametitle{Gera��o Procedural X Entradas do Usu�rio}
%\begin{itemize}
%	\item Quanto menor o n�mero de entradas do usu�rio, maior o n�vel de gera��o procedural.
%\end{itemize}
%\begin{center}
%\includegraphics[width=0.5\linewidth]{img/node}
%\end{center}
%\end{frame}



\begin{frame}\frametitle{Vantagens da gera��o procedural} 
\begin{itemize}
	\item Flexibilidade: alterando os par�metros do algoritmo, � poss�vel gerar um grande n�mero de modelos.
	\item Espa�o: n�o h� necessidade de um grande espa�o em disco, j� que tudo ser� ditado por algoritmos.
\end{itemize}
\end{frame}

\begin{frame}\frametitle{Exemplos}
\begin{columns}
	\begin{column}{5cm}
	\begin{itemize}
		\item \alert<1>{.kkrieger\\} \only<1>{\scriptsize Praticamente tudo gerado proceduralmente}
		\item \alert<2>{Elite (1984)\\} \only<2>{\scriptsize Oito gal�xias, 256 planetas.}
		\item \alert<3>{SpeedTree\\} \only<3>{\scriptsize �rvores geradas proceduralmente.}
	\end{itemize}
	\vspace{3cm} 
	\end{column}
	\begin{column}{5cm}
	\begin{overprint}
		\includegraphics<1>[width=1.0\linewidth]{img/kkrieger}
		\includegraphics<2>[width=1.0\linewidth]{img/elite}
		\includegraphics<3>[width=1.0\linewidth]{img/speedtree}
	\end{overprint}
	\end{column}
\end{columns}
\end{frame}


\section{Metodologia}

\subsection{Metodologia}

\begin{frame}\frametitle{Metodologia} 
\begin{itemize}
	\item Livro \emph{Texturing and Modeling: A Procedural Approach} \cite{livro}.
	\item Estudo das melhoras formas de reduzir o gasta com mem�ria atrav�s de estruturas de dados do OpenGL.
	\item Implementa��o do arcabou�o.
\end{itemize}
\end{frame}

\section{Proposta}

\subsection{Proposta}

\begin{frame}\frametitle{Proposta}
\begin{itemize}
	\item O objetivo deste trabalho � construir um arcabou�o para a cria��o de terrenos proceduralmente em tempo real e que permita a inser��o de modelos pelo usu�rio, como, por exemplo, na forma de mapas de altura.
	\item �reas gen�ricas ser�o geradas proceduralmente, e �reas que necessitam de maior detalhe, ser�o visualizadas por meio de mapas de altura.
\end{itemize}
\end{frame}

\begin{frame}\frametitle{Proposta}
\begin{itemize}
	\item O arcabou�o est� sendo constru�do de forma que possa suportar terrenos criados de diversas maneiras.
	\begin{itemize}
		\item Arquivos com mapas de altura
		\item Fault Formation
		\item Perlin Noise (Ru�do de Perlin)
		\item Fbm
	\end{itemize}
\end{itemize}
\begin{center}
\includegraphics[width=0.5\linewidth]{img/node}
\end{center}
\end{frame}

\subsection{Gera��o procedural dos terrenos}

\begin{frame}\frametitle{Ru�do de Perlin}
\begin{columns}
	\begin{column}{5cm}
	\begin{itemize}
		\item O ru�do � usado para simular estruturas naturais, como n�vens, texturas de �rvores, e terrenos.
	\end{itemize}
	\vspace{3cm} 
	\end{column}
	\begin{column}{5cm}
	\begin{overprint}
		\includegraphics[width=1.0\linewidth]{img/perlin}
	\end{overprint}
	\end{column}
\end{columns}
\end{frame}


\subsection{Grafo de cena}

\begin{frame}\frametitle{Terrenos}
\begin{itemize}
	\item Cada terreno � um quadrado.
	\item � poss�vel variar quantos terrenos s�o visualizados e gerados.
\end{itemize}
\begin{center}
\includegraphics[width=0.7\linewidth]{img/grafo_cena1}
\end{center}
\end{frame}

\begin{frame}\frametitle{Grafo de cena - Organiza��o}
\begin{columns}
	\begin{column}{5cm}
	\begin{itemize}
		\item Um grafo para armazenar os terrenos que ser�o renderizados.
		\item Um nodo do grafo de cena aponta para os oito terrenos vizinhos.
		\item Quando a c�mera muda de terreno, novos terrenos s�o gerados.
		\item Os v�rtices s�o armazenados em uma estrutura de dados VBO.
	\end{itemize}
	\vspace{3cm}
	\end{column}
	\begin{column}{5cm}
	\begin{overprint}
		\includegraphics<1>[width=1.0\linewidth]{img/grafo_cena2}
		\includegraphics<2>[width=1.0\linewidth]{img/grafo_cena3}
	\end{overprint}
	\end{column}
\end{columns}
\end{frame}





\chapter{RESULTADOS E DISCUSS�O}
\label{resultados}

Os conhecimentos adquiridos ao longo desse trabalho permitiram a cria��o de um sistema capaz de gerar terrenos procedurais tanto na \emph{GPU} quanto na \emph{CPU}, e tamb�m permite a navega��o do usu�rio por tal terreno. Na Se��o \ref{sistema} ser� apresentado tal sistema, bem como deci��es e detalhes de implementa��o. Finalmente, na Se��o \ref{terrenosgerados} ser� apresentado alguns exemplos de terrenos gerados proceduralmente.



\section{O Sistema Implementado}
\label{sistema}


O sistema implementado neste trabalho teve como principal objetivo permitir a gera��o procedural de terrenos tanto na \emph{GPU} quanto na \emph{CPU}. A Figura \ref{fig:bibliotecas} apresenta as camadas do sistema, destacando as bibliotecas utilizadas (como � explicado a seguir).

\begin{figure}[H]
	\center{\includegraphics[width=0.2\linewidth]{img/bibliotecas.png}}
	\caption{\label{fig:bibliotecas} Camadas do sistema.}
\end{figure}

\begin{itemize}
	\item {\bf OpenGL}: \emph{API} gr�fica utilizada para a renderiza��o gr�fica.
	\item {\bf glew}: Biblioteca para carregamento de extens�o do \emph{OpenGL}.
	\item {\bf glfw}: Biblioteca que facilita o tratamento de entradas e tamb�m cria��o de janelas.
	\item {\bf ftgl}: Biblioteca para a renderiza��o de textos.
	\item {\bf FreeType}: Biblioteca para renderiza��o de textos (depend�ncia do \emph{ftgl}).
	\item {\bf WindowMng}: Camada respons�vel por criar a tela e tratar os eventos de entrada.
	\item {\bf ProcTerrain}: Camada respons�vel por gerar e exibir os terrenos.
\end{itemize}

As camadas \emph{WindowMng} e \emph{ProcTerrain} foram implementadas neste trabalho. O \emph{WindowMng} tem como proposito simular a camada de um aplicativo gr�fico gen�rico (\emph{game}, simulador, etc.); desta forma, o sistema poder� ser posteriormente adaptado para funcionar em conjunto com outros aplicativos que possam vir a ser desenvolvidos.

A Figura \ref{fig:arquitetura} apresenta em detalhes os m�dulos presentes nas camadas \emph{WindowMng} e \emph{ProcTerrain}. A seguir, uma explica��o sobre cada um dos m�dulos.

\begin{figure}[H]
	\center{\includegraphics[width=0.5\linewidth]{img/arquitetura.png}}
	\caption{\label{fig:arquitetura} Diagrama com as principais classes do sistema implementado.}
\end{figure}

\begin{itemize}
	\item {\bf WindowMng}: Respons�vel por simular um aplicativo gr�fico gen�rico, e chamar os devidos \emph{callbacks} do pacote \emph{ProcTerrain}.
	\item {\bf Camera}: M�dulo que implementa uma c�mera controlada pelo jogador e navegando pelo mundo.
	\item {\bf TerrainMng}: M�dulo respons�vel por gerar e controlar os terrenos.
	\item {\bf SquareNode}: Nodo que representa uma fatia do terreno.
	\item {\bf HeightMap}, {\bf HeightMapCPU} e {\bf HeightMapGPU}: M�dulos que implementam os mapas de altura dos terrenos gerados na \emph{CPU} ou na \emph{GPU}.
	\item {\bf GenerationShader}: \emph{Shader} respons�vel pela gera��o dos terrenos.
	\item {\bf RenderingShader}: \emph{Shader} respons�vel pela renderiza��o dos terrenos.
\end{itemize}

A seguir, nas Se��es \ref{geracao} e \ref{visualizacao} ser�o apresentados os detalhes de implementa��o da gera��o e visualiza��o do terreno.


\section{Gera��o do Terreno}
\label{geracao}
Nesta se��o, ser� abordado a implementa��o da gera��o de terrenos, tanto na \emph{GPU}, quanto na \emph{CPU}. Os dois t�m, em comum, o algoritmo usado para a gera��o (\emph{Ridged multifractal noise}).


\subsection{Gera��o do Terreno na \emph{GPU}}
Toda a gera��o dos terrenos na \emph{GPU} � feita atrav�s de um \emph{fragment shader}. Como toda computa��o de \emph{shaders} s� pode ser aplicada com base em geometrias ou texturas, foi preciso renderizar um quadrado. Entra a� ent�o o \sigla{FBO}{Frame Buffer Object}, extens�o do \emph{OpenGL} que permite criar \emph{Frame Buffers} e realizar renderiza��es \emph{off-screen} (que n�o s�o exibidas na tela). O quadrado � renderizado em um novo \emph{frame buffer}, e o \emph{shader} de gera��o procedural � aplicado sobre ele. O resultado pode ser visto na Figura \ref{fig:resultados:heightmap}.

\begin{figure}[H]
	\center{\includegraphics[width=0.5\linewidth]{img/caps/heightmap.png}}
	\caption{\label{fig:resultados:heightmap} Mapa de altura gerado pelo \emph{shader}.}
\end{figure}

\subsection{Gera��o do Terreno na \emph{CPU}}
A gera��o na \emph{CPU} � feita de maneira tradicional. Uma lista de v�rtices � criada e, para cada v�rtice, � calculado a sua altura de acordo com o algoritmo \emph{Ridged multifractal noise}.



\section{Visualiza��o do Terreno}
\label{visualizacao}
Com o mapa de altura gerado, o pr�ximo passo � exibir o terreno para o usu�rio. Da mesma forma, aqui tamb�m ser� exposto alguns detalhes espec�ficos da visualiza��o do terreno gerado na \emph{GPU}, na Se��o \ref{visualizacaogpu}. Antes, por�m, � necess�rio detalhar fun��es comuns tanto � visualiza��o de terrenos gerados na \emph{CPU} quanto na \emph{GPU}.

O passo inicial � a gera��o de uma malha (conjunto de v�rtices) de tamanho pr�-determinado, como mostra a Figura \ref{fig:malha}

\begin{figure}[H]
	\center{\includegraphics[width=0.5\linewidth]{img/caps/malha.png}}
	\caption{\label{fig:malha} Malha inicial para visualiza��o dos terrenos.}
\end{figure}

A malha � gerada de tal forma que um n�mero maior de v�rtices est� concentrado no centro. Quanto maior a dist�ncia, menor o n�mero de v�rtices presentes. Isto propicia uma maneira r�pida e f�cil de implementar um n�vel de detalhamento (quanto maior a dist�ncia do centro, menor ser� a necessidade de se renderizar o terreno em alta fidelidade).

Como a malha � gerada apenas uma �nica vez (no in�cio da execu��o), n�o � preciso criar repetidas malhas a medida que o jogador percorre o terreno. Apenas os mapas de altura s�o trocados, como mostra a Figura \ref{fig:texturas}

\begin{figure}[H]
	\center{\includegraphics[width=0.5\linewidth]{img/texturas.png}}
	\caption{\label{fig:texturas} Movimenta��o da c�mera do nodo (1,1), na esquerda, para o nodo (2, 1), na direita.}
\end{figure}

Na Figura \ref{fig:texturas} � poss�vel notar o deslocamento dos mapas de textura quando a c�mera se move do nodo (1,1) para o nodo (2,1). Este m�todo, poss�vel somente para terrenos gerados na \emph{GPU}, diminuiu a necessidade de implementa��o de um algoritmo de n�vel de detalhe mais robusto. Al�m disso, como sabemos o n�mero de v�rtices antecipadamente, a performance do aplicativo tem uma menor chance de sofrer quedas bruscas de rendimento.


\subsection{Visualiza��o do Terreno Gerado na \emph{GPU}}
\label{visualizacaogpu}
Para a visualiza��o do terreno gerado na \emph{GPU}, um \emph{vertex shader} l� a altura do mapa de altura gerado e desloca a posi��o de \emph{z} do v�rtice correspondente na malha.

Um aspecto importante � que a gera��o do mapa de altura j� fornece os valores das normais de cada v�rtice (necess�rios para o c�lculo da luz).




\section{Terrenos gerados}
\label{terrenosgerados}
Nesta Se��o, ser� apresentado uma s�rie de imagens com o resultado das gera��es procedurais na \emph{GPU}.

A Figura \ref{fig:heightmap2} mostra a renderiza��o de uma cena apenas com a textura do mapa de altura aplicado em um quadrado.
\begin{figure}[H]
	\center{\includegraphics[width=0.5\linewidth]{img/caps/heightmap2.png}}
	\caption{\label{fig:heightmap2} Mapa de altura aplicado a um quadrado.}
\end{figure}


A Figura \ref{fig:deslocamento} mostra o mesmo mapa de altura, mas agora com o deslocamento dos v�rtices em no eixo \emph{z}.
\begin{figure}[H]
	\center{\includegraphics[width=0.5\linewidth]{img/caps/deslocamento.png}}
	\caption{\label{fig:deslocamento} Mapa de altura aplicado a um quadrado, deslocando a altura.}
\end{figure}

A Figura \ref{fig:blend} mostra agora a renderiza��o da malha com texturas para simular o grama, pedras, neve, etc.
\begin{figure}[H]
	\center{\includegraphics[width=0.5\linewidth]{img/caps/blend.png}}
	\caption{\label{fig:blend} Mapa de altura aplicado a um quadrado, deslocando a altura, e com texturas.}
\end{figure}

A Figura \ref{fig:luz} mostra o resultado final, agora com a aplica��o de ilumina��o.
\begin{figure}[H]
	\center{\includegraphics[width=0.5\linewidth]{img/caps/luz.png}}
	\caption{\label{fig:luz} Mapa de altura aplicado a um quadrado, deslocando a altura, com texturas, e ilumina��o.}
\end{figure}


\section{Testes}
Alguns testes foram feitos para avaliar a velocidade de gera��o dos terrenos com a altera��o de alguns par�metros, utilizando tanto a \emph{GPU} quanto a \emph{CPU}. Eles foram executados em um \emph{Core 2 Duo E7400}, com 2GB de mem�ria \emph{RAM} e placa de v�deo \emph{ATI Radeon HD 4850} com 512MB de mem�ria \emph{RAM}, e \emph{driver} vers�o 8.612. As tabelas com os tempos e as conclus�es dos testes s�o apresentadas a seguir.


A Se��o \ref{testeGeracao} mostra um teste com a gera��o de 100 terrenos proceduralmente.

\subsection{Tempos de Gera��o do Terreno}
\label{testeGeracao}
Neste teste foi medido o tempo m�dio gasto com a gera��o de terrenos tanto na \emph{GPU} quanto na \emph{CPU}. Para cada n�mero de \emph{octaves} (4, 8, 12, 16), foi gerado 100 terrenos, e medido o tempo gasto. O mapa de altura utilizado possui tamanho 512 x 512. Os resultados obtidos est�o na tabela \ref{tabela:geracao}.


\begin{table}[H]
	\begin{center}
		\begin{tabular}{|c|c|c|}
			\hline
			\emph{Octaves} & GPU & CPU \\
			\hline
			4 & 28,7236 & 46,2948\\
			\hline
			8 & 28,7432 & 92,4299\\
			\hline
			16 & 28,7887 & 187,103\\
			\hline
			32 & 30,1095 & 370,841\\
			\hline
		\end{tabular}
		\caption{Tempo m�dia (em ms) de gera��o dos terrenos, com n�mero vari�vel de \emph{octaves}}
		\label{tabela:geracao}
	\end{center}
\end{table}


A Figura \ref{fig:geracao} apresenta os tempos anteriores.
\begin{figure}[H]
	\center{\includegraphics[width=0.5\linewidth]{img/tempoGeracao.png}}
	\caption{\label{fig:geracao} Gr�fico do tempo m�dio (em ms) de gera��o dos terrenos, com n�mero vari�vel de \emph{octaves}.}
\end{figure}


Como pode ser visto, o tempo de gera��o do terreno na \emph{GPU} permanece quase constante, enquanto a gera��o na \emph{CPU} tem um comportamento praticamente linear.



\chapter{CONCLUS�ES E TRABALHOS FUTUROS}
\label{conclusao}

%O trabalho desenvolvido em POC I teve como objetivo principal estabelecer uma base para ser desenvolvida posteriormente em POC II, e tamb�m como uma forma de aprendizado de t�cnicas procedurais. A intera��o do usu�rio com o terreno gerado ser� aprofundada ainda mais, permitindo a inser��o n�o s� de mapas de altura, mas tamb�m de modelos tridimensionais, como, por exemplo, casas.

%Um ponto a ser otimizado � a c�pia dos v�rtices para as estruturas VBO. Dessa forma, ser� poss�vel diminuir o tempo total gasto com a gera��o procedural dos terrenos, e suavizar as transi��es entre terrenos. Outro ponto que pode ser abordado � a visualiza��o dos terrenos na medida em que eles s�o gerados; no caso do ru�do de Perlin, a visualiza��o poderia ser a cada \emph{octave} gerado. Terrenos mais longe da c�mera poderiam tamb�m ser gerados com um n�mero menor de \emph{octaves}. Al�m disso, o processamento em paralelo pode ser explorado, um dos temas abordados no paradigma apresentado em \cite{carlucio}.

%O uso de texturas tamb�m ser� aprofundado, possivelmente com o uso de sombras pr�-calculadas \cite{slopelighting}. Outra quest�o a ser desenvolvida em POC II � uma \emph{interface} gr�fica mais atrativa, e uma forma eficiente de se armazenar as informa��es inseridas pelo usu�rio, possivelmente em arquivos \emph{Extensible Markup Language} (\sigla{XML}{Extensible Markup Language}).

Os resultados obtidos na gera��o procedural de terrenos na \emph{GPU} mostram o poder de processamento das placas gr�ficas, em rela��o �s \emph{CPU}. Este trabalho, por�m, n�o implementou uma alternativa \emph{multi-core} para a gera��o procedural na \emph{CPU}. Isto poderia diminuir o tempo gasto na gera��o. Outro aspecto que pode ser abordado no futuro � o escalonamento entre \emph{GPU} e \emph{CPU}, dependendo do n�vel de ociosidade de cada processador.

Um ponto a ser melhorado na gera��o procedural � quanto � heterogeneidade do terreno. Apesar de apresentar um resultado satisfat�rio para um local limitado, quando se olha o terreno como um todo, percebe-se que h� uma semelhan�a muito grande entre os \emph{patchs}, diminuindo assim a ilus�o de estar andando em algo realmente infinito.

O sistema foi implementado sempre tendo em mente a sua utiliza��o acoplado a outras aplicativos. Dessa forma, adot�-lo em um simulador ou \emph{game} demandaria pouco esfor�o, desde que o aplicativo utilize \emph{OpenGL}.


\section{Bibliografia}
\begin{tiny}
\nocite{*}
\bibliographystyle{unsrt}
\bibliography{apresentacao_monografia}
\end{tiny}


\begin{frame}
\Large D�vidas?
\end{frame}




\end{document}
