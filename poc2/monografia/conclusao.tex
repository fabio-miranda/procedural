\chapter{CONCLUS�ES E TRABALHOS FUTUROS}
\label{conclusao}

%O trabalho desenvolvido em POC I teve como objetivo principal estabelecer uma base para ser desenvolvida posteriormente em POC II, e tamb�m como uma forma de aprendizado de t�cnicas procedurais. A intera��o do usu�rio com o terreno gerado ser� aprofundada ainda mais, permitindo a inser��o n�o s� de mapas de altura, mas tamb�m de modelos tridimensionais, como, por exemplo, casas.

%Um ponto a ser otimizado � a c�pia dos v�rtices para as estruturas VBO. Dessa forma, ser� poss�vel diminuir o tempo total gasto com a gera��o procedural dos terrenos, e suavizar as transi��es entre terrenos. Outro ponto que pode ser abordado � a visualiza��o dos terrenos na medida em que eles s�o gerados; no caso do ru�do de Perlin, a visualiza��o poderia ser a cada \emph{octave} gerado. Terrenos mais longe da c�mera poderiam tamb�m ser gerados com um n�mero menor de \emph{octaves}. Al�m disso, o processamento em paralelo pode ser explorado, um dos temas abordados no paradigma apresentado em \cite{carlucio}.

%O uso de texturas tamb�m ser� aprofundado, possivelmente com o uso de sombras pr�-calculadas \cite{slopelighting}. Outra quest�o a ser desenvolvida em POC II � uma \emph{interface} gr�fica mais atrativa, e uma forma eficiente de se armazenar as informa��es inseridas pelo usu�rio, possivelmente em arquivos \emph{Extensible Markup Language} (\sigla{XML}{Extensible Markup Language}).

Os resultados obtidos na gera��o procedural de terrenos na \emph{GPU} mostram o poder de processamento das placas gr�ficas, em rela��o �s \emph{CPU}. 


