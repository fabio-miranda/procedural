\section{Testes}
\label{testes}

Para verificar a efici�ncia das gera��es na CPU e GPU, foi feito um teste que consiste na navega��o por um trajeto constante durante 60 segundos. O computador utilizado foi um \emph{Core 2 Duo E7400}, com 2GB de mem�ria \emph{RAM} e placa de v�deo \emph{ATI Radeon HD 4850} com 512MB de mem�ria \emph{RAM}, e \emph{driver} vers�o 8.612. As seguintes configura��es foram utilizadas para a gera��o procedural: 8 \emph{octaves}, \emph{lacunarity} igual a 2.5, \emph{gain} igual a 0.5,  \emph{offset} como 0.9 e n�mero de vizinhos referente ao \emph{patch} central igual a 2.

Quatro testes foram executados, cada um com uma configura��o diferente de utiliza��o da CPU e GPU. O primeiro teste foi executado somente na GPU ($\alpha$ = 1); o segundo somente na CPU ($\alpha$ = 0) e utilizando uma �nica \emph{thread} para todo o programa; os testes tr�s e quatro foram executados tanto na GPU quanto na CPU, sendo que o teste tr�s utilizou uma �nica \emph{thread} e o teste tr�s utilizou um esquema \emph{multithread} para a gera��o na CPU.

Como � poss�vel ver atrav�s da Figura \ref{fig:fps}, a utiliza��o da gera��o apenas na GPU se mostrou a mais vantajosa. No computador utilizado, nenhuma queda no \emph{frame rate} foi percebida. A gera��o totalmente na CPU (utilizando v�rias \emph{threads}) se mostrou tamb�m bastante eficiente. Pequenas quedas no \emph{frame rate} ocorreram devido a transfer�ncia das texturas da mem�ria principal para a mem�ria da placa de v�deo. Como era de se esperar, a gera��o na CPU com apenas uma \emph{thread} se mostrou impratic�vel do ponto de vista da fluidez da navega��o.