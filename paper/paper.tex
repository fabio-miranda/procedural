\documentclass[a4paper]{sbgames}               % final
%\usepackage[scaled=.92]{helvet}

\usepackage[brazil]{babel}
\usepackage[latin1]{inputenc}


\usepackage{times}
\usepackage{graphicx}

%% use this for zero \parindent and non-zero \parskip, intelligently.
\usepackage{parskip}

%% the 'caption' package provides a nicer-looking replacement
\usepackage[labelfont=bf,textfont=it]{caption}

\usepackage{url}

%% Paper title.
\title{Um Sistema para Gera��o Procedural de Terrenos Pseudo-Infinitos em Tempo-Real Utilizando GPU e CPU}

%% Author and Affiliation (multiple authors). Use: and between authors

%\author{Name1 A. Surname1\\ Digital Games Lab 
%        \and Name2 B. Surname2\\ Name3 C. Surname3\\ ZZZZ University
%        \and Name4 D. Surname4\\ Farwest Research Center 
%}
%\contactinfo{\{name1,name3\}@xxx.yyyy.yyy \\
%             *name2@zzzz.vvvv.vvv
%}

%% Keywords that describe your work.
\keywords{modelagem procedural; gpu; gpgpu; programa��o paralela}

%% Start of the paper
% Attention: As you need to insert EPS images in Postscript, 
% you need to insert PDF images into PDFs. 
% In the text, extensions cancbe omitted (latex use .eps, pdflatex get .pdf) 
% To convert them: epstopdf myimage.eps
\begin{document}

\teaser{
  \includegraphics[width=.8\linewidth]{img/teaser.png}
  \caption{Terreno gerado proceduralmente.}
  \label{fig:teaser}
}

%% The ``\maketitle'' command must be the first command after the
%% ``\begin{document}'' command. It prepares and prints the title block.

\maketitle

%% Abstract section.

\begin{abstract}
O r�pido crescimento do poder de processamento das placas gr�ficas fez com que diversas tarefas migrassem da CPU para a GPU. Por�m, as unidades de processamento gr�fico podem ser vistas como aliadas da CPU, e n�o rivais. Este trabalho prop�e um sistema \emph{multithread} que utiliza tanto a CPU quanto a GPU para minimizar o tempo gasto com a gera��o procedural de terrenos e permitir uma navega��o fluida atrav�s de um mundo pseudo-infinito gerado proceduralmente.

Ao final, uma compara��o � feita com base em testes com tr�s modelos de gera��o (apenas CPU, apenas GPU, GPU e CPU), com o objetivo de expor suas vantagens e desvantagens.
\end{abstract}

%% The ``\keywordlist'' command prints out the keywords.
\keywordlist
\contactlist

\chapter{INTRODU��O}
\label{introducao}

\section{Vis�o geral}
A gera��o procedural de modelos � uma �rea da Ci�ncia da Computa��o que prop�e que modelos gr�ficos tridimensionais (representa��o em pol�gonos de algum objeto) possam ser gerados atrav�s de rotinas e algoritmos. Tal t�cnica vem se tornando bastante popular nos �ltimos tempos, tendo em vista que, com o crescimento da ind�stria do entretenimento, h� uma necessidade de se construir modelos cada vez maiores e com um grande n�vel de detalhe. A t�cnica de gera��o procedural vem ent�o como uma alternativa �utiliza��o do trabalho de artistas e modeladores na cria��o de modelos tridimensionais.

Outro fato tamb�m muito relevante atualmente s�o as (\emph{Graphics Processing Unit} (\sigla{GPU}{Graphics Processing Unit})). Estes microprocessadores, incorporados �s placas de v�deo atuais, s�o especializados em processar gr�ficos. O avan�o da ind�stria de \emph{games} fez com que as \emph{GPUs} se tornassem cada vez mais r�pidas, tornando-as atraente para outras �reas da computa��o.

\section{Objetivo, justificativa e motiva��o}
O objetivo deste trabalho � permitir a cria��o de terrenos proceduralmente em tempo real, aproveitando o poder de processamento da \emph{GPU}. O trabalho pode ser dividido em duas vertentes: visualiza��o de terrenos e sua gera��o proceduralmente.


O primeiro aspecto (a visualiza��o) � um problema muito estudado no campo da computa��o. O modelo de um terreno � algo que pode demandar um n�mero extremamente alto de tri�ngulos. O processo de gerar a imagem a partir desse modelo (ou renderiza��o) em tempo real fica inviabilizado nos computadores atuais. Faz-se ent�o necess�ria a utiliza��o de t�cnicas que limitam e minimizam o n�mero de tri�ngulos a serem desenhados na tela. Entra a� o uso de \emph{culling} e n�veis de detalhe dos modelos.

A segunda parte, gera��o de terrenos proceduralmente, busca, atrav�s de algoritmos, criar terrenos realistas e que possam ser utilizados em jogos eletr�nicos, simuladores, ou qualquer tipo de aplica��o que necessita de um ambiente virtual tridimensional. Os principais benef�cios dessa gera��o s�o:

\begin{itemize}
	\item Compress�o de dados: todos os modelos s�o criados por algoritmos, e n�o h� a necessidade de se armazenar os dados dos modelos gerados, diminuindo assim o tempo necess�rio para a transmiss�o do conte�do por uma rede, por exemplo.
	\item Conte�do gerado pelo usu�rio de forma f�cil: com a gera��o procedural, o usu�rio n�o precisa de um grande conhecimento ou ent�o a contrata��o de artistas para poder construir modelos 3D. Uma interface amig�vel e alguns ajustes de par�metros s�o o bastante para gerar modelos interessantes.
	\item Produtividade: quanto menor o n�mero de entradas um sistema procedural possuir, menor o trabalho necess�rio para criar modelos.
\end{itemize}


Este trabalho ir� implementar solu��es para a gera��o procedural de terrenos na \emph{GPU} e tamb�m na \emph{CPU}, com o objetivo de comparar a efici�ncia e velocidade das duas abordagens.


\section{Organiza��o}
Este trabalho est� organizado da seguinte maneira. O Cap�tulo \ref{introducao} apresenta uma breve introdu��o e objetivos. O Cap�tulo \ref{referencialTeorico} apresenta o referencial te�rico pertinente a este trabalho. O Cap�tulo \ref{metodologia} mostra a metodologia envolvida. O Cap�tulo \ref{resultados} apresenta os resultados obtidos e, finalmente, o Cap�tulo \ref{conclusao} mostra a conclus�o do trabalho e propostas para trabalhos futuros.



\section{Trabalhos relacionados}
\label{trabalhosRelacionados}

Uma das t�cnicas de modelagem procedural de terrenos � o ru�do Perlin \cite{perlinNoise}, uma fun��o pseudo-aleat�ria que, dado uma entrada (posi��o), retorna um valor que possui uma suave transi��o com os seus vizinhos. Em \cite{improvedPerlinNoise} foi apresentado um ru�do Perlin otimizado, que buscou tornar o ru�do mais amig�vel �s novas arquiteturas (GPUs), melhorar as propriedades visuais e introduzir uma �nica vers�o do ru�do que retornaria os mesmos valores independentemente da plataforma de \emph{hardware} ou \emph{software}.


Em \cite{proceduralApproach} s�o apresentados alguns algoritmos que fazem uso do ru�do Perlin e que s�o capazes de gerar terrenos de uma forma significativamente realista. Podemos citar o algoritmo \emph{fBm}, \emph{heterogenous terrain}, \emph{hybrid multifractal} e \emph{ridged multifractal}, sendo que este �ltimo foi o algoritmo utilizado neste trabalho.

Em \cite{carlucio}, os autores apresentam um paradigma para a modelagem procedural (terrenos, vegeta��o, etc.) utilizando v�rias \emph{threads}. Uma implementa��o � proposta utilizando apenas as unidades de processamento dispon�veis na CPU.

A gera��o procedural utilizando a GPU foi explorada em \cite{generatingComplex} e \cite{Schneider:2006:FractalTerrain}. O primeiro trabalho, faz uso de \emph{geometry shaders} e est� limitado �s placas de v�deo com suporte a DirectX 10. O segundo trabalho, mais abrangente quanto as placas de v�deo suportadas, gera os terrenos na GPU com o uso de algoritmos multifractais (semelhante ao que � proposto aqui). Nenhum dos dois trabalhos, por�m, faz uma compara��o entre implementa��es de gera��o de terrenos utilizando a CPU e a GPU, e tamb�m n�o buscam uma plataforma que utilize as duas arquiteturas.

\section{Contribui��es}
\label{contribuicoes}

O trabalho apresentado aqui possui as seguintes contribui��es:

\begin{itemize}
	\item Propor uma plataforma de gera��o procedural de terrenos que fa�a uso tanto da CPU quanto da GPU.
	\item Fazer um estudo quanto aos benef�cios da gera��o procedural na CPU, na GPU, e em uma arquitetura GPU/CPU.
\end{itemize}
\section{Conceitos b�sicos}

\subsection{Ru�do Perlin}


\subsection{Fractais}


\subsection{Ridged Multifractal Noise}

\section{Proposta}

\subsection{Proposta}

\begin{frame}\frametitle{Proposta}
\begin{itemize}
	\item O objetivo deste trabalho � construir um arcabou�o para a cria��o de terrenos proceduralmente em tempo real e que permita a inser��o de modelos pelo usu�rio, como, por exemplo, na forma de mapas de altura.
	\item �reas gen�ricas ser�o geradas proceduralmente, e �reas que necessitam de maior detalhe, ser�o visualizadas por meio de mapas de altura.
\end{itemize}
\end{frame}

\begin{frame}\frametitle{Proposta}
\begin{itemize}
	\item O arcabou�o est� sendo constru�do de forma que possa suportar terrenos criados de diversas maneiras.
	\begin{itemize}
		\item Arquivos com mapas de altura
		\item Fault Formation
		\item Perlin Noise (Ru�do de Perlin)
		\item Fbm
	\end{itemize}
\end{itemize}
\begin{center}
\includegraphics[width=0.5\linewidth]{img/node}
\end{center}
\end{frame}

\subsection{Gera��o procedural dos terrenos}

\begin{frame}\frametitle{Ru�do de Perlin}
\begin{columns}
	\begin{column}{5cm}
	\begin{itemize}
		\item O ru�do � usado para simular estruturas naturais, como n�vens, texturas de �rvores, e terrenos.
	\end{itemize}
	\vspace{3cm} 
	\end{column}
	\begin{column}{5cm}
	\begin{overprint}
		\includegraphics[width=1.0\linewidth]{img/perlin}
	\end{overprint}
	\end{column}
\end{columns}
\end{frame}


\subsection{Grafo de cena}

\begin{frame}\frametitle{Terrenos}
\begin{itemize}
	\item Cada terreno � um quadrado.
	\item � poss�vel variar quantos terrenos s�o visualizados e gerados.
\end{itemize}
\begin{center}
\includegraphics[width=0.7\linewidth]{img/grafo_cena1}
\end{center}
\end{frame}

\begin{frame}\frametitle{Grafo de cena - Organiza��o}
\begin{columns}
	\begin{column}{5cm}
	\begin{itemize}
		\item Um grafo para armazenar os terrenos que ser�o renderizados.
		\item Um nodo do grafo de cena aponta para os oito terrenos vizinhos.
		\item Quando a c�mera muda de terreno, novos terrenos s�o gerados.
		\item Os v�rtices s�o armazenados em uma estrutura de dados VBO.
	\end{itemize}
	\vspace{3cm}
	\end{column}
	\begin{column}{5cm}
	\begin{overprint}
		\includegraphics<1>[width=1.0\linewidth]{img/grafo_cena2}
		\includegraphics<2>[width=1.0\linewidth]{img/grafo_cena3}
	\end{overprint}
	\end{column}
\end{columns}
\end{frame}





\section{Implementa��o}
\label{implementacao}

A Figura \ref{fig:bibliotecas} apresenta as camadas do sistema, bem como as bibliotecas utilizadas.

\begin{figure}[h]
	\center{\includegraphics[width=0.6\linewidth]{img/bibliotecas.png}}
	\caption{\label{fig:bibliotecas} Camadas do sistema.}
\end{figure}

A camada \emph{WindowMng} tem como prop�sito simular a um aplicativo gr�fico gen�rico (\emph{game}, simulador, etc.); desta forma, o sistema poder� ser posteriormente adaptado para funcionar em conjunto com outros aplicativos que possam ser desenvolvidos (ou acoplado a uma \emph{engine}).

A Figura \ref{fig:arquitetura} apresenta em detalhes os m�dulos presentes nas camadas \emph{WindowMng} e \emph{ProcTerrain}. A seguir, uma explica��o sobre cada um dos m�dulos.

\begin{figure}[h]
	\center{\includegraphics[width=0.6\linewidth]{img/arquitetura.png}}
	\caption{\label{fig:arquitetura} Diagrama com as principais classes do sistema implementado.}
\end{figure}

\begin{itemize}
	\item {\bf WindowMng}: Respons�vel por simular um aplicativo gr�fico gen�rico, e chamar os devidos \emph{callbacks} do pacote \emph{ProcTerrain}.
	\item {\bf Camera}: M�dulo que implementa uma c�mera controlada pelo jogador e navegando pelo mundo.
	\item {\bf TerrainMng}: M�dulo respons�vel por gerar e controlar os terrenos.
	\item {\bf SquareNode}: Nodo que representa uma fatia (\emph{patch}) do terreno.
	\item {\bf HeightMap}: M�dulo que implementa os mapas de altura dos terrenos gerados na CPU ou na GPU.
	\item {\bf GenerationShader}: \emph{Shader} respons�vel pela gera��o dos terrenos.
	\item {\bf RenderingShader}: \emph{Shader} respons�vel pela renderiza��o dos terrenos.
\end{itemize}

\section{Testes}
\label{testes}

Para verificar a efici�ncia das gera��es na CPU e GPU, foi feito um teste que consiste na navega��o por um trajeto constante durante 60 segundos. O computador utilizado foi um \emph{Core 2 Duo E7400}, com 2GB de mem�ria \emph{RAM} e placa de v�deo \emph{ATI Radeon HD 4850} com 512MB de mem�ria \emph{RAM}, e \emph{driver} vers�o 8.612. As seguintes configura��es foram utilizadas para a gera��o procedural: 8 \emph{octaves}, \emph{lacunarity} igual a 2.5, \emph{gain} igual a 0.5,  \emph{offset} como 0.9 e n�mero de vizinhos referente ao \emph{patch} central igual a 2.

Quatro testes foram executados, cada um com uma configura��o diferente de utiliza��o da CPU e GPU. O primeiro teste foi executado somente na GPU ($\alpha$ = 1); o segundo somente na CPU ($\alpha$ = 0) e utilizando uma �nica \emph{thread} para todo o programa; os testes tr�s e quatro foram executados tanto na GPU quanto na CPU, sendo que o teste tr�s utilizou uma �nica \emph{thread} e o teste tr�s utilizou um esquema \emph{multithread} para a gera��o na CPU.

Como � poss�vel ver atrav�s da Figura \ref{fig:fps}, a utiliza��o da gera��o apenas na GPU se mostrou a mais vantajosa. No computador utilizado, nenhuma queda no \emph{frame rate} foi percebida. A gera��o totalmente na CPU (utilizando v�rias \emph{threads}) se mostrou tamb�m bastante eficiente. Pequenas quedas no \emph{frame rate} ocorreram devido a transfer�ncia das texturas da mem�ria principal para a mem�ria da placa de v�deo. Como era de se esperar, a gera��o na CPU com apenas uma \emph{thread} se mostrou impratic�vel do ponto de vista da fluidez da navega��o.
\section{Conclus�o e Trabalhos Futuros}
\label{conclusao}

Os resultados obtidos na gera��o procedural de terrenos na GPU mostram o poder de processamento das placas gr�ficas em rela��o �s CPU. A op��o de se gerar nas duas arquiteturas mostra-se promissora, principalmente considerando a utiliza��o do sistema acoplado a um \emph{game} ou simulador. Como haver� inevitavelmente outras tarefas sendo executadas (\emph{path-finding}, sombras, HDR), a possibilidade de se migrar a carga de trabalho envolvida na gera��o procedural pode ser bastante vantajosa.

Como trabalho futuro, espera-se encontrar uma estrat�gia eficiente para o c�lculo de $\alpha$. Dessa forma, a gera��o procedural poder� ser balanceada automaticamente entre as diferentes arquiteturas utilizadas (CPU e GPU).

O sistema foi implementado sempre tendo em mente a sua utiliza��o acoplada a outras aplicativos. Assim, adot�-lo em um \emph{game} ou simulador demandaria pouco esfor�o.

\bibliographystyle{sbgames}
\bibliography{paper}

%\begin{figure*}
	\center{\includegraphics[width=0.58\linewidth]{img/fps_single.pdf}}
	\caption{\label{fig:fpssingle} Gr�fico com o FPS na navega��o pelo mundo durante 300 segundos, utilizando uma �nica \emph{thread} na CPU.}
\end{figure*}

\begin{figure*}
	\center{\includegraphics[width=0.58\linewidth]{img/fps_multi.pdf}}
	\caption{\label{fig:fpsmulti} Gr�fico com o FPS na navega��o pelo mundo durante 300 segundos, utilizando m�ltiplas \emph{threads} na CPU.}
\end{figure*}

%\begin{figure*}[t]
%	\center{\includegraphics[width=1.0\linewidth]{img/fps.pdf}}
%	\caption{\label{fig:fps} Gr�fico com o FPS na navega��o pelo mundo durante 60 segundos.}
%\end{figure*}

%\begin{figure}[h]
%	\center{\includegraphics[width=1.0\linewidth]{img/caps/heightmap.png}}
%	\caption{\label{fig:heightmap} Mapa de altura aplicado a um quadrado.}
%\end{figure}

%\begin{figure}[h]
%	\center{\includegraphics[width=1.0\linewidth]{img/caps/deslocamento.png}}
%	\caption{\label{fig:deslocamento} Mapa de altura aplicado a um quadrado, deslocando a altura.}
%\end{figure}

%\begin{figure}[h]
%	\center{\includegraphics[width=1.0\linewidth]{img/caps/blend.png}}
%	\caption{\label{fig:blend} Mapa de altura aplicado a um quadrado, deslocando a altura, e com texturas.}
%\end{figure}

%\begin{figure}[h]
%	\center{\includegraphics[width=1.0\linewidth]{img/caps/luz.png}}
%	\caption{\label{fig:luz} Mapa de altura aplicado a um quadrado, deslocando a altura, com texturas, e %ilumina��o.}
%\end{figure}

\end{document}
